\setchapterabstract{}
\chapter{Vectors}
\vspace{-1.5cm}

{\chaptoc\noindent\begin{minipage}[inner sep=0,outer sep=0]{0.9\linewidth}\section{Vector Space}\end{minipage}}

    You start with a set $V$ and this set $V$ has two operations:
        \begin{itemize}
            \item An internal operation denoted by $+: V \times V \rightarrow V$.
                The \(\times\) symbol is read "Cartesian product" and it means that the operation takes two elements from the set $V$ and returns an element in the set $V$.
                \textit{A priori}, "+" might not be commutative.
                You read it with two elements $A$ and $B$ and it gives you back an element in the set $V$.
                If you feed it $B$ and $A$, it gives you back an element (which could be different) in the set $V$.
            \item \textbf{Scalar Product:} An outer operation denoted by $\cdot: \mathbb{R} \times V \rightarrow V$
                This operation takes an element from the set $\mathbb{R}$ and an element from the set $V$ and returns an element in the set $V$.
                This operation is commutative.
                It is an outer operation as you feed it with an element in $V$, a real number (which is outside $V$) and it gives you back an element in $V$.
                We call the real number a scalar.
        \end{itemize}

    \Definition{
        It is a set $V$ endowed with two operations: the sum and the scalar product which satisfy the following 8 properties (4 and 4). 4 Properties only apply to the sum, and other 4 apply to both scalar and sum.
    }{Vector Space}


\subsection{4 Properties of the Sum}

    Those properties are:
    \begin{enumerate}
        \item \textbf{Associativity}: it tells us how to sum multiple elements; let's start with 3 elements $X$, $Y$ and $Z$ in the set $V$, denoted ad $X,Y,Z \in V$.
            \begin{itemize}
                \item $X + (Y + Z) = (X + Y) + Z$
            \end{itemize}
            The order of the elements cannot change unless we introduce the following property, i.e. commutativity.
        \item \textbf{Existence of the Neutral Element}: there exists an element in the set $V$ such that when you sum it with any element in the set $V$, you get back the same element.
            \begin{itemize}
                \item $\exists 0 \in V: \forall X \in V \ \ \ \ X + 0 = 0 + X = X$
            \end{itemize}
            It is just a name, it is not the number 0. So far, we can just say that it exists but we cannot assert its uniqueness. (spoiler: it is indeed unique)
        \item \textbf{Existence of the Inverse Element}: for every element in the set $V$, there exists an element in the set $V$ such that when you sum them, you get back the neutral element.
            \begin{itemize}
                \item $\forall X \in V, \exists Y \in V \ \ \ \ x+y = y+x = 0$
            \end{itemize}
        \item \textbf{Commutativity}: the order of the elements does not matter.
            \begin{itemize}
                \item $\forall x,y \in V \ \ \ \  X + Y = Y + X$
            \end{itemize}
    \end{enumerate}

    \Remark{
        There is a difference between 
        \[\exists 0 \in V \ \ \ \ \forall x \in V\]
        \[\forall x \in V \ \ \ \ \exists y \in V\]
    }

    \Example{
        The above remark is equivalent to saying:
        \begin{itemize}
            \item There exists an ideal shoe brand for everybody
                e.g. Adidas is the best brand for all of us
            \item For every person, there exists an ideal shoe brand
                e.g. Adidas is the best brand for me, Nike is the best brand for you (i.e. the brand might be person-specific)
        \end{itemize}
    }


    \subsection{4 Properties of the Sum and Scalar Product}

    \begin{enumerate}
        \item[a.] \textbf{Distributivity of the scalar product over the sum}: it tells us how to distribute the scalar product over the sum of two elements.
            \begin{itemize}
                \item $\forall x,y, \in V, \ \forall \alpha \in \mathbb{R} \ \ \ \  \alpha (x+y) = \alpha x + \alpha y $
            \end{itemize}
        \item[b.] \textbf{Questo mi manca}: take 2 scalars and 2 elements of the set $V$.
            \begin{itemize}
                \item $ \forall \alpha , \beta \in \mathbb{R}, \ \forall x \in V \ \ \ \ (\alpha + \beta) x = \alpha x + \beta x $
            \end{itemize}
        \item[c.] \textbf{Associativity of the scalar product}:
            \begin{itemize}
                \item $\forall \alpha, \beta \in \mathbb{R}, \ \forall x \in V \ \ \ \ (\alpha \beta) x = \alpha (\beta x) $
            \end{itemize}
        \item[d.] \textbf{Existence of the Neutral Element for the scalar product}:
            \begin{itemize}
                \item $\exists 1 \in \mathbb{R} \ \ \ \ \forall x \in V \ \ \ \ 1x = x$
            \end{itemize}
    \end{enumerate}
        

    \Note{
        The order might not exactly reflect the order of the properties in the book/handout.
    }

    \Remark{
        When you are given a vector space, you are given the set $V$, the two operations and the 4+4 properties.
        You don't know anything more about anything else.
    }

    \Note{
        We will see more examples on thursday 12th .
    }
    
    fine 09-09-2024